\chapter{Аналитическая часть}

\section{Расстояние Левенштейна}

\textbf{Расстояние Левенштейна} -- между двумя строками в теории информации и компьютерной лингвистике -- это минимальное количество операций вставки одного символа, удаления одного символа и замены одного символа на другой, необходимых для превращения одной строки в другую.~\cite{itmo-levenstein}

\subsection{Рекурсивная формула расстояния Левенштейна}

Пусть существует две последовательности символов $S_1$ и $S_2$ длины N и M соответсвенно. Ввёдем функцию $D(i, j)$, равную расстоянию левенштейна между подстроками $S_1[1...i]$ и $S_2[1...j]$.

Пусть
\begin{itemize}
    \item совпадение символа обозначено как M;
    \item вставка символа обозначена как A;
    \item удаление символа обозначено как D;
    \item замена символа обозначена как R.
\end{itemize}

Тогда расстояние Левенштейна для двух строк может быть выражено следующей функцией:

\begin{equation}
D_1(s1[1..i], s2[1..j]) = 
\begin{cases}
    0, & i=0,\ j=0, \quad M \\
    
    j, & i=0,\ j>0, \quad I \\
    
    i, & i>0,\ j=0, \quad D \\
    
    min\begin{pmatrix}
        D_1(s1[1..i], s2[1..j-1]) + 1, & I \\
        
        D_1(s1[1..i-1], s2[1..j]) + 1, & D \\
        
        D_1(s1[1..i-1], s2[1..j-1]) + \\ % Тут пришлось порвать строчку, чтобы уместить контент на экране
        
        \begin{cases}
            0, & s1_i = s2_i \quad M \\
            
            1, & else \quad\qquad R
        \end{cases}
    \end{pmatrix} & \text{иначе}
\end{cases}
\label{eq:rec}
\end{equation}

Результат функции - это минимальное расстояние Левенштейна от строки 1 до строки s2, представленных в виде последовательности символов. Базой рекурсии являются ситуации, когда хотя бы одна из строк пустая. В остальных случаях выбирается наименьшее значение из трёх случаев:
\begin{itemize}
    \item вставка символа в s1 со штрафом 1;
    \item удаление символа из s1 со штрафом 1;
    \item либо совпадение символа со штрафом 0, либо замена со штрафом 1.
\end{itemize}

Под понятием штраф имеется в виду ``цена'' редакторской операции, учитываемая в итоговом значении расстояния Левенштейна

\subsection{Расстояние Левенштейна с кешированием}
Рекурсивная форма может оказаться малоэффективной при больших значениях N и M, так как в ней могут постоянно пересчитываться значения $D(i, j)$ для одних и тех же аргументов. Для устранения потенциального недостатка используют итерационный алгоритм, который сохраняет промежуточные значения в виде матрицы размерности $(N+1)x(M+1)$.

Значения в ячейке [i, j] матрицы содержат значение $D(i, j)$, то есть расстояние левенштейна между подстроками $S_1[1...i]$ и $S_2[1...j]$.
При этом нулевая строка и нулевой столбец матрицы заполняются слева-направо и сверху-вниз от 0 до соотвествующей размерности,  так как приведение пустой строки к любой строке длины, скажем, i требует i операций вставки.

Далее алгоритм построчно заполняет матрицу по формуле~\ref{eq:rec}, при этом вместо просчитывания предыдущих значений D используются значения из матрицы в соотвествующих ячейках.

Данный алгоритм может работать эффективнее по времени, однако за счёт хранения дополнительной матрицы может тратить больше памяти. Чтобы это оптимизировать, на каждой итерации хранят только текущую строку матрицы и предыдущую, так как остальные не нужны для просчёта, однако я не делал этого в своих реализациях алгоритмов.

\section{Расстояние Дамерау-Левенштейна}

\textbf{Расстояние Дамерау-Левенштейна} -- между двумя строками, состоящими из конечного числа символов -- это минимальное число операций вставки, удаления, замены одного символа и транспозиции двух соседних символов, необходимых для перевода одной строки в другую.~\cite{itmo-damerau-levenstein}

\subsection{Рекурсивная формула расстояния Дамерау-Левенштейна}

Введём также как и для расстояния Левенштейна функцию $D(s1, s2)$. 

Пусть операция транспозиции обозначена как T

Тогда, с учётом транспозиции формула~\ref{eq:rec} преобразутся в формулу~\ref{eq:damerau}

\begin{equation}
\label{eq:damerau}
D_1(s1[1..i], s2[1..j]) = 
    \begin{cases}
        0, & i=0,\ j=0,\ M \\
        
        j, & i=0,\ j>0,\ I \\
        
        i, & i>0,\ j=0,\ D \\
        
        min\begin{pmatrix}
            D_1(s1[1..i], s2[1..j-1]) + 1, & I \\
            
            D_1(s1[1..i-1], s2[1..j]) + 1, & D \\
            
            D_1(s1[1..i-1], s2[1..j-1]) + \\ % Тут пришлось порвать строчку, чтобы уместить контент на экране
            
            \begin{cases}
                0, & s1_i = s2_i \quad M \\
                
                1, & else \quad\qquad R
            \end{cases}
        \end{pmatrix} & \begin{aligned}
            &i, j > 1, \\ 
            &S_1[i] = S_2[j - 1], \\
            &S_1[i - 1] = S_2[j]\\
        \end{aligned} \\

\\

        min\begin{pmatrix}
            D_1(s1[1..i], s2[1..j-1]) + 1, & I \\
            
            D_1(s1[1..i-1], s2[1..j]) + 1, & D \\
            
            D_1(s1[1..i-1], s2[1..j-1]) + \\ % Тут пришлось порвать строчку, чтобы уместить контент на экране
            
            \begin{cases}
                0, & s1_i = s2_i \quad M \\
                
                1, & else \quad\qquad R
            \end{cases}
        \end{pmatrix} & \text{иначе}
    \end{cases}
\end{equation}

Фактическое отличие от алгоритма расстояния Левенштейна в том, что при j > 1 и i > 1, а также при условии что соседние символы в строках равны ``накрест'', может быть проведена операция транспозиции. Транспозиция считается редакторской операцией со сложностью 1, когда в обычном алгоритме поиска расстояния Левенштейна, такая операция имела бы сложность 2

\subsection{Расстояние Дамерау-Левенштейна с кешированием}
Также как и в случае расстояния Левенштейна, рекурсивный алгоритм может быть малоэффективным, поэтому имеет место быть итерационный алгоритм с кешированием, используя матрицу, либо текущую и последние две строки матрицы. Алгоритм аналогичен алгоритму расстояния, за исключением того, что при равенстве накрест двух соседних символов строк и значению i>1 и j>1 при расчёте очередной ячейки нужно проверить ещё и вариант с транспозицией.

\section*{Вывод}
В результате аналитического раздела были рассмотрены расстояния Левенштейна и Дамерау-Левенштейна, а также их рекурсивные формулы, итерационные алгоритмы с кешированием матрицами и отдельными строками матриц.

\clearpage