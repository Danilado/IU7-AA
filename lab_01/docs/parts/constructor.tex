\chapter{Конструкторская часть}

\section{Требования к программному обеспечению}

К разрабатываемой программе предъявлен ряд требований:

\begin{itemize}
	\item на вход подаются две строки -- s1 и s2;
	\item на выход подаётся целое число;
	\item заглавные и строчные символы считаются разными;
	\item возможность обработки строк как на кириллице так и на латинице;
	\item должна быть возможность замера процессорного времени программы;
	\item должна быть возможность вывода графиков и таблиц замеров процессорного времени и памяти.
\end{itemize}

Был введён вспомогательный алгоритм инициализации таблицы (кэша) для алгоритмов с кэшем.

Для остальных алгоритов всегда подразумевается, что на вход подаётся две строки, являющиеся последовательностями символов с именами s1 и s2, а на выходе ожидается целое число

\section{Разработка алгоритмов}

Алгоритм инициализации таблицы (кэша) изображённый на рисунке~\ref{fig:fill}

Рекурсивный алгоритм нахождения расстояния Левенштейна изображён на рисунке~\ref{fig:rec}

Алгоритм нахождения расстояния Левенштейна с кэшем изображён на рисунке~\ref{fig:iter}

Алгоритм нахождения расстояния Дамерау-Левенштейна с кэшем изображён на рисунке~\ref{fig:damerau}

\begin{figure}
	\centering
	\includesvg[width=0.2\textwidth]{table_fill}
	\caption{Алгоритм инициализации кэш-таблицы}
	\label{fig:fill}
\end{figure}

\clearpage

\begin{figure}
	\centering
	\includesvg[width=0.9\textwidth]{levenstein_rec}
	\caption{Рекурсивный алгоритм нахождения расстояния Левенштейна}
	\label{fig:rec}
\end{figure}

\clearpage

\begin{figure}
	\centering
	\includesvg[width=0.73\textwidth]{levenstein_iter}
	\caption{Алгоритм нахождения расстояния Левенштейна с кешэм}
	\label{fig:iter}
\end{figure}

\clearpage

\begin{figure}
	\centering
	\includesvg[width=0.9\textwidth]{damerau_levenstein}
	\caption{Алгоритм нахождения расстояния Дамерау-Левенштейна с кешэм}
	\label{fig:damerau}
\end{figure}

\clearpage

\section{Вывод}

В результате конструкторской части были определены требования к ПО, а также построены схемы алгоритмов рекурсивного поиска расстояния Левенштейна, итерационного поиска с кешэм и итерационного алгоритма поиска расстояния Дамерау-Левенштейна с кешэм.

\clearpage