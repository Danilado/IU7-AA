\ssr{ВВЕДЕНИЕ}

\textbf{Расстояние Левенштейна} -- минимальное количество редакторских операций для превращения одной строки в другую.

Под редакторской операцией понимается одна из операций следующего списка:
\begin{itemize}
    \item вставка символа;
    \item удаление символа;
    \item замена символа;
    \item другие операции в вариациях алгоритма. Например, Транспозиция для алгоритма Дамерау-Левенштейна.
\end{itemize}

Несколько сфер применения алгоритма Расстояния Левенштейна:

\begin{itemize}
\item компьютерная лингвистика (например, исправление ошибок пользовательского ввода текста);
\item биоинформатика (например, Анализ иммунитета).
\end{itemize}

Целью данной лабораторной работы является изучение методов динамического программирования на примере алгоритмов Левенштейна и Дамерау-Левенштейна.

Задачами данной лабораторной являются:
\begin{enumerate}
  \item рассмотрение алгоритмов Левенштейна и Дамерау-Левенштейна нахождения расстояния между строками;
  \item разработка рекурсивного алгоритма расстояния Левенштейна, а также применение методов динамического программирования для разработки алгоритмов расстояний Левенштейна и Дамерау-Левенштейна с использованием кеширования;
  \item экспериментальное подтверждение различий во временной эффективности рекурсивной и итерациной реализаций алгоритма определения расстояния Левенштейна с помощью замеров процессорного времени и используемой памяти при
  выполнении алгоритмов на варьирующихся длинах строк.
\end{enumerate}
\newpage
\clearpage