\ssr{ЗАКЛЮЧЕНИЕ}

По результатам исследования оказалось, что рекурсивная реализация алгоритма нахождения расстояния Левенштейна существенно проигрывает итерационной по временным характеристикам, однако в отношении ёмкостных характеристик ситуация противоположная - в большинстве случаев, рекурсивная реализация эффективнее итерационной.

В ходе данной лабораторной работы были выполнены следующие задачи:
\begin{enumerate}
	\item были рассмотрены алгоритмы Левенштейна и Дамерау-Левенштейна нахождения расстояния между строками;
	\item были разработаны:
	\begin{itemize}
		\item рекурсивный алгоритм нахождения расстояния Левенштейна;
		\item алгоритм нахождения расстояния Левенштейна с кэшированием;
		\item алгоритм нахождения расстояния Дамерау-Левенштейна с кэшированием.
	\end{itemize} 
	\item были экспериментально подтверждены различия во временной эффективности рекурсивной и итерациной реализаций алгоритма определения расстояния Левенштейна с помощью замеров процессорного времени и используемой памяти при выполнении алгоритмов на варьирующихся длинах строк.
\end{enumerate}

Для разработки алгоритмов нахождения расстояния Левенштейна и \\ Дамерау-Левенштейна с кэшированием, были изучены и применены методы динамического программирования.

Цель и задачи лабораторной работы были выполнены.
