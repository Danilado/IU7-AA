\chapter{Исследовательская часть}

\section{Технические характеристики}

Технические характеристики устройства, на котором проводились замеры:

\begin{itemize}
	\item операционная система: EndeavourOS x86\_64;
	\item процессор: 13th Gen Intel(R) Core(TM) i53500H (16) С частотой 4.70 ГГц;
	\item оперативная память: 16 ГБ с частотой 5200 МГц.
\end{itemize}

При проведении замеров ноутбук был подключен к сети и переведён в режим максимальной производительности. Запущена была только система и Jupyter Notebook

Для сравнения характеристик алгоритмов, на вход алгоритмам подаются две случайные строки одинаковой длины. Длины строк приведены в соответствующих таблицах. 

\section{Временные характеристики}
В таблице~\ref{tbl:time_table} приведены временные характеристики, полученные в результате замеров времени всех трёх алгоритмов.

Для каждой длины замер проводился 50 раз, при этом для каждой итерации создавались 2 случайные строки заданной длины и использовались как аргументы к алгоритмам. Итоговое время взято как среднее арифметическое всех полученных замеров.

Рекурсивный алгоритм нахождения расстояния Левенштейна работает очень долго уже при длине строки равной 12, поэтому замеры времени для строк б\emph{о}льшей длины проводились только на алгоритмах с кэшированием. знак --- в таблице значит отсутствие данных.

\begin{longtable}[]{|c|r|r|r|}
\caption{Таблица зависимости времени обработки строк от длины строк}
\label{tbl:time_table}
\\
\hline
Длина & Рек. алгоритм (нс) & Итер. алгоритм (нс) &  Дамерау-Левенштейна (нс) \\ \hline
1 & 1 545 & 1 893 & 1 414 \\ \hline
2& 2 981& 2 282&  10 693 \\ \hline
3& 7 267& 3 279&3 333 \\ \hline
4&40 702& 4 412&4 807 \\ \hline
5& 211 776& 5 936&6 818 \\ \hline
6&1 142 547& 8 848&9 103 \\ \hline
7&7 755 973&16 490&  14 253 \\ \hline
8&  28 786 587&19 901&  15 577 \\ \hline
9& 132 354 547&20 192&  18 618 \\ \hline
10&868 123 744&23 266&  22 291 \\ \hline
11&  3 564 314 047&29 834&  26 889 \\ \hline
12& 27 435 792 490&33 648&  31 348 \\ \hline
13 & --- &33 332&  35 821 \\ \hline
14 & --- &35 886&  40 974 \\ \hline
15 & --- &39 146&  45 935 \\ \hline
16 & --- &45 502&  52 078 \\ \hline
17 & --- &49 880&  59 384 \\ \hline
18 & --- &55 522&  66 703 \\ \hline
19 & --- &61 686&  73 413 \\ \hline
20 & --- &68 260&  78 292 \\ \hline
21 & --- &74 541& 124 237 \\ \hline
22 & --- &81 820&  95 959 \\ \hline
23 & --- &87 478& 102 833 \\ \hline
24 & --- &94 959& 111 450 \\ \hline
25 & --- & 102 708& 122 344 \\ \hline
26 & --- & 110 758& 132 019 \\ \hline
27 & --- & 119 178& 140 822 \\ \hline
28 & --- & 128 560& 151 071 \\ \hline
29 & --- & 136 739& 160 726 \\ \hline
30 & --- & 147 645& 175 105 \\ \hline
31 & --- & 155 039& 182 966 \\ \hline
32 & --- & 167 214& 197 397 \\ \hline
33 & --- & 176 132& 210 144 \\ \hline
34 & --- & 188 196& 221 512 \\ \hline
35 & --- & 203 662& 240 423 \\ \hline
36 & --- & 211 826& 248 477 \\ \hline
37 & --- & 222 059& 263 062 \\ \hline
38 & --- & 232 068& 276 508 \\ \hline
39 & --- & 246 967& 293 653 \\ \hline
40 & --- & 259 247& 305 639 \\ \hline
41 & --- & 274 170& 324 889 \\ \hline
42 & --- & 286 795& 340 290 \\ \hline
43 & --- & 297 447& 351 554 \\ \hline
44 & --- & 312 274& 372 037 \\ \hline
45 & --- & 332 470& 393 392 \\ \hline
46 & --- & 342 904& 406 540 \\ \hline
47 & --- & 359 710& 426 905 \\ \hline
48 & --- & 372 541& 442 847 \\ \hline
49 & --- & 385 227& 457 590 \\ \hline
50 & --- & 402 756& 477 728 \\ \hline
\end{longtable}


Полученные замеры также визуализированы на графике~\ref{fig:time_graph}:

\begin{figure}[h!]
	\centering
    \includesvg[width=0.8\textwidth]{time_graph.svg}
	\caption{График зависимости времени работы алгоритмов от длины строк}
	\label{fig:time_graph}
\end{figure}

\begin{figure}[h!]
	\centering
    \includesvg[width=0.8\textwidth]{time_graph_norec.svg}
	\caption{График зависимости времени работы алгоритмов от длины строк без рекурсивного алгоритма Левенштейна}
	\label{fig:time_graph_norec}
\end{figure}

Как видно из таблицы~\ref{tbl:time_table} и графика~\ref{fig:time_graph}, рекурсивный алгоритм существенно уступает итерационному по временным характеристикам.

Чтобы наглядно сравнить временные характеристики итерационных алгоритмов нахождения расстояния Левенштейна и Дамерау-Левенштейна, я также построил график~\ref{fig:time_graph_norec}.

Из таблицы~\ref{tbl:time_table} и графика~\ref{fig:time_graph_norec} также видно, что итерационный алгоритм Дамерау-Левенштейна стабильно хуже по временным характеристикам, чем итерациный алгоритм Левенштейна.

\subsection{Замеры с платы}

Замеры проводились на плате со следующими техническими характеристиками:
\begin{itemize}
	\item модель -- STM32F767
	\item процессор -- Arm-Cortex-M7 с частотой 216 МГц
	\item операцивная память -- 512 КБ
\end{itemize}

Результаты замеров можно увидеть в таблице~\ref{tbl:time_table_circuit}

\begin{longtable}[]{|c|r|r|r|}
\caption{Таблица зависимости времени обработки строк от длины строк на плате}
\label{tbl:time_table_circuit}
\\
\hline
Длина & Рек. алгоритм (мс) & Итер. алгоритм (мс) &  Дамерау-Левенштейна (мс) \\ \hline
1 & 0 		& 0 & 0 \\ \hline
2 & 0 		& 1 & 0 \\ \hline
3 & 2 		& 0 & 1 \\ \hline
4 & 11 		& 1 & 0 \\ \hline
5 & 61 		& 1 & 1 \\ \hline
6 & 184		& 1 & 2 \\ \hline
7 & 1 527 	& 2 & 2 \\ \hline
8 & 8 359 	& 2 & 2 \\ \hline
\end{longtable}

\section{Емкостные характеристики}

В таблице~\ref{tbl:mem_table} приведены емкостные характеристики, полученные в результате замеров памяти рекурсивного и итерационного алгоритмов с помощью модуля tracemalloc.
Так как для рекурсивной реализации размер выделяемой памяти существенно зависит от глубины рекурсии, то есть от входных строк, то для каждой длины строки для рекурсивного алгоритма алгоритм был проведен 50 раз на случайных строках. 
Для итерационного алгоритма выделяемая память не зависит от входных строк(при одинаковой длине), так как размер матрицы кеша зависит только от размеров строк. 
За характеристику памяти взято среднее арифметическое значение использованной памяти.

\

\begin{longtable}{|c|c|c|}
\caption{Таблица зависимости объёма использованной памяти при обработке строк от их длины}
\label{tbl:mem_table}
\\
\hline
Длина строки & Рек. Алгоритм (Б) & Итер. Алгоритм (Б) \\ \hline
1	&	47		&	192 \\ \hline 
2	&	48		&	232 \\ \hline 
3	&	253		&	288 \\ \hline 
4	&	251		&	392 \\ \hline 
5	&	398		&	480 \\ \hline 
6	&	477		&	584 \\ \hline 
7	&	568		&	704 \\ \hline 
8	&	664		&	904 \\ \hline 
9	&	832		&	1056 \\ \hline 
10	&	1180	&	1224 \\ \hline 
\end{longtable}

Также зависимость из таблицы отражена на графике~\ref{fig:mem_graph}.

Как видно из графика~\ref{fig:mem_graph} и таблицы~\ref{tbl:mem_table}, в среднем, рекурсивная реализация выигрывает итерационную по ёмкостным характеристикам.

\begin{figure}[h]
	\centering
	\includesvg[width=0.9\textwidth]{mem_graph.svg}
	\caption{График зависимости требуемой памяти для работы алгоритмов от длины входных строк}
	\label{fig:mem_graph}
\end{figure}

\pagebreak

\section{Вывод}

В данном разделе было проведено исследование временных и емкостных характеристик алгоритмов рекурсивного поиска расстояния Левенштейна и итерационного поиска расстояние Левенштейна с кешем.

По результатам исследования оказалось, что рекурсивная реализация существенно проигрывает итерационной по времени, однако в отношении ёмкостных характеристик ситуация противоположная - в большинстве случаев, рекурсивная реализация эффективнее итерационной.

\clearpage