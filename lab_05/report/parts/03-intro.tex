\ssr{ВВЕДЕНИЕ}

\textbf{Параллелизм} --- это возможность выполнения нескольких процессов одновременно~\cite{par-citation}.

С точки зрения программирования, параллелизм --- это возможность систем производить вычисления одновременно.

В тех ситуациях, где это возможно и обоснованно, использование параллельных вычислений может привести к улучшению временных характеристик программ.

Такими случаями могут быть, например, задачи в которых нужно произвести сложную обработку нескольких независимых сущностей, или задачи, где возможно производить вычисления в ожидании ввода/вывода для других элементов программы~\cite{tanenbaum}.

В современных системах параллельные вычисления разделяют на два типа~\cite{tanenbaum}:
\begin{itemize}
  \item потоковая параллельность;
  \item параллельность, основанная на процессах.
\end{itemize}

Целью данной работы является разработка программного обеспечения, обрабатывающего веб-страницы онлайн ресурса chiefs.kz.

Для достижения этой цели нужно выполнить следующие задачи:
\begin{itemize}
  \item рассмотреть структуру страниц с рецептами ресурса chiefs.kz;
  \item разработать ПО, выполняющее обработку веб-страницы и запись данных в БД;
  \item разработать ПО, выполняющее эту задачу методом конвейерной обработки;
  \item реализовать разработанное ПО;
  \item проанализировать полученную реализацию по временным характеристикам.
\end{itemize}

\clearpage