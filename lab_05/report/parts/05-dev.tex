\chapter{Разработка ПО}

Для реализации алгоритмов был выбран язык программирования c++ стандарта 14882 (c++20)~\cite{cpp}. Этот язык достаточен для выполнения работы, так как его стандартная библиотека предлагает средства для работы с нативными потоками.

Для работы с базами данных была использована библиотека SQLite3~\cite{sqlite}. Она предоставляет такие функции для работы с базой данных, как:
\begin{itemize}
    \item создание БД;
    \item создание таблиц;
    \item поддержка SQL запросов;
\end{itemize}

Для извлечения элементов с html страниц была использована библиотека Gumbo~\cite{gumbo}. Эта библиотека предоставляет возможность поиска определённых html-элементов с учётом различных параметров, а также даёт возможность извлечения данных с них.

Принцип конвейерной обработки был реализован с использованием стандартных функций для работы с потоками языка c++, однако для поддержания потоконезависимой очереди была использована библиотека OneTBB~\cite{onetbb}, а конкретно --- её модуль <<concurrent\_queue>>

Структуры данных, использующиеся для конвейерной обработки:

Структура задачи представлена в листинге~\ref{lst:task}.

\begin{lstlisting}[caption={Структура задачи},label={lst:task}]
    
class Task {
    public:
      int id;
      int stage;
      std::chrono::time_point<std::chrono::high_resolution_clock> timestamp;
      std::vector<std::chrono::microseconds> times;
    
      std::string filename;
      std::string data;
    
      std::string title;
      std::string image;
      std::string ingredients;
      std::vector<std::string> steps;
    
    public:
      Task() {}
      Task(int id, const std::string &filename) : id(id), filename(filename) {
        timestamp = std::chrono::high_resolution_clock::now();
      }
    
      std::chrono::microseconds nextStage() {
        std::chrono::time_point<std::chrono::high_resolution_clock> t2 =
            std::chrono::high_resolution_clock::now();
    
        times.push_back(
            std::chrono::duration_cast<std::chrono::microseconds>(t2 - timestamp));
    
        stage++;
    
        timestamp = std::chrono::high_resolution_clock::now();
    
        return times.back();
      }
    };    
\end{lstlisting}

Класс, управляющий задачами представлен в листинге~\ref{lst:taskmgr}.

\begin{lstlisting}[caption={Класс, управляющий задачами},label={lst:taskmgr}]
class TaskManager {
private:
  tbb::concurrent_queue<Task> stage1Queue;
  tbb::concurrent_queue<Task> stage2Queue;
  tbb::concurrent_queue<Task> stage3Queue;
  tbb::concurrent_vector<Task> finished;
  std::atomic<int> taskCounter{0};

  std::atomic<bool> running = false;

  Logger logger;
  DatabaseManager db;

  void logTask(const Task &task, const std::string &message,
               const std::chrono::microseconds &time);

  void readFileStage();
  void processDataStage();
  void writeToDBStage();

public:
  TaskManager();
  void run();
  void stop();
  void push(Task t);
  const tbb::concurrent_vector<Task> &getFinished() { return finished; }
};
\end{lstlisting}

Задачи попадают в очередь первой стадии при использовании функции push у управляющего задачами.

Обработка начинается при использовании функции run и заканчивается при использовании функции stop (однако при использовании stop процесс будет заблокирован до завершения обработки всех процессов, находящихся в очереди).

Реализация функции run и функций каждого этапа обработки представлены в листинге~\ref{lst:taskmgrrun}.

\begin{lstlisting}[caption={Реализация функции run и функций каждого этапа обработки},label={lst:taskmgrrun}]
void TaskManager::run() {
    running = true;
    std::thread reader(&TaskManager::readFileStage, this);
    std::thread processor(&TaskManager::processDataStage, this);
    std::thread writer(&TaskManager::writeToDBStage, this);
}

void TaskManager::readFileStage() {
  while (running || taskCounter) {
    Task task;
    if (stage1Queue.try_pop(task)) {
      auto duration = task.nextStage();
      logTask(task, "started step 1", duration);

      task.data = Reader::read(task.filename);

      duration = task.nextStage();
      stage2Queue.push(task);
      logTask(task, "finished step 1", duration);
    }
  }
}

void TaskManager::processDataStage() {
  while (running || taskCounter) {
    Task task;
    if (stage2Queue.try_pop(task)) {
      auto duration = task.nextStage();
      logTask(task, "started step 2", duration);

      Parser p(task.data);
      task.title = p.getTitle();
      task.image = p.getImage();
      task.ingredients = p.getIngredients();
      task.steps = p.getSteps();

      duration = task.nextStage();
      stage3Queue.push(task);
      logTask(task, "finished step 2", duration);
    }
  }
}

void TaskManager::writeToDBStage() {
  while (running || taskCounter) {
    Task task;
    if (stage3Queue.try_pop(task)) {
      auto duration = task.nextStage();
      logTask(task, "started step 3", duration);

      db.insertRecipe(task.id, task.title, task.ingredients, task.steps,
                      task.image);

      duration = task.nextStage();
      finished.push_back(task);
      taskCounter--;
      logTask(task, "finished step 3", duration);
    }
  }
}

\end{lstlisting}

Таким образом, создаётся 3 потока, выполняющих обработку каждого из трёх этапов соответственно.

Задачи перемещаются по очередям соответствующих этапов, пока их обработка не будет завершена.
