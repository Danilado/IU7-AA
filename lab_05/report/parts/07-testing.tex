\chapter{Тестирование ПО}

Тестирование программы проводилось посредством загрузки 1024 html-файлов.
При этом отслеживалось отсутствие ошибок, а также вручную были проверены полученные в БД записи обо всех четырёх страницах, доступных на сайте.

На листинге~\ref{lst:log} представлен фрагмент лога, сформированного конвейером в процессе обработки задач.

\begin{lstlisting}[caption={фрагмент лога программы},label={lst:log}]
Task 1023 finished step 1 in    [34      microseconds.]
Task 1024 started step 1 in     [38559   microseconds.]
Task 1024 finished step 1 in    [33      microseconds.]
Task 31 finished step 2 in      [1066    microseconds.]
Task 32 started step 2 in       [36726   microseconds.]
Task 25 finished step 3 in      [1227    microseconds.]
Task 26 started step 3 in       [6108    microseconds.]
Task 32 finished step 2 in      [1193    microseconds.]
Task 33 started step 2 in       [37940   microseconds.]
Task 26 finished step 3 in      [1258    microseconds.]
Task 27 started step 3 in       [6291    microseconds.]
\end{lstlisting}

Как видно из фрагмента лога, первый этап обработки для всех 1024 задач был завершён уже тогда, когда обработку вторым этапом проходила только 31-я задача.
После окончания обработки 31 задачи, конвейер сразу приступил к обработке задачи с номером 32 --- следующей в очереди. Аналогичное поведение наблюдается и с третьим этапом (задачами 25, 26).

\section{Описание исследования}

Было проведено исследование времени обработки и времени ожидания задач в очередях каждого из этапов обработки.

Результаты тестирования приведены в таблице~\ref{tbl:results}.

\begin{table}[h!]
    \caption{Таблица времени обработки задач на разных этапах}
    \label{tbl:results}
    \begin{tabular}{|l|r|}    
        \hline
        Этап & Среднее время этапа (мкс) \\ \hline
        Полная обработка & 1308225 \\ \hline
        Ожидание в очереди первого этапа &91305 \\ \hline
        Обработка на первом этапе & 107 \\ \hline
        Ожидание в очереди второго этапа & 1212228 \\ \hline
        Обработка на втором этапе & 2256 \\ \hline
        Ожидание в очереди третьего этапа & 446 \\ \hline
        Обработка на третьем этапе & 1881 \\ \hline
    \end{tabular}
\end{table}

\section{Вывод}

В результате измерений было обнаружено, что при обработке 1024 страниц с ресурса chiefs.kz, второй этап оказался более трудоёмким, чем первый примерно в 22 раза. Именно этим обусловлено сравнительно большое время простоя в очереди второго этапа. Третий же этап обработки занимал меньше времени, чем второй, поэтому блокировки в очереди на третий этап не происходило.