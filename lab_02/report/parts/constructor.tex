\chapter{Конструкторская часть}

\section{Требования к программному обеспечению}

К разрабатываемой программе предъявлен ряд требований:

\begin{itemize}
	\item на вход подаются две матрицы A и B;
	\item на выход подаётся матрица C равная произведению матрицы A на матрицу B;
	\item под матрицей в программе понимается двумерный массив чисел;
	\item индексация в массивах начинается с 0;
	\item в случае если произведение матрицы A на матрицу B не определено (количество столбцов первой матрицы и рядов второй не совпадает), должно быть выведено сообщение об ошибке;
\end{itemize}

\section{Разработка алгоритмов}

В алгоритмах подразумевается, что матрица C уже инициализирована и требуется только её заполнение.

Классический алгоритм умножения матриц изображён на рисунке~\ref{fig:alg_classic}

Алгоритм умножения матриц Копперсмита-Винограда изображён на рисунках~\ref{fig:alg_winograd} и~\ref{fig:alg_winograd_sub}

\begin{figure}[h!]
	% \centering
	\includesvg[height=0.9\textheight]{classic.svg}
	\caption{Классический алгоритм умножения матриц}
	\label{fig:alg_classic}
\end{figure}

\begin{figure}[h!]
	% \centering
	\includesvg[width=0.9\textwidth]{wino.svg}
	\caption{Алгоритм умножения матриц Копперсмита-Винограда}
	\label{fig:alg_winograd}
\end{figure} 

\begin{figure}[h!]
	\centering
	\includesvg[height=0.9\textheight]{winosub.svg}
	\caption{Вторая часть алгоритма умножения матриц Копперсмита-Винограда}
	\label{fig:alg_winograd_sub}
\end{figure} 

\section{Вывод}

В результате конструкторской части были определены требования к ПО, а также разработаны алгоритмы классического умножения матриц и алгоритм Копперсмита-Винограда.

\clearpage