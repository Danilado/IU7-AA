\chapter{Технологическая часть}
\section{Средства разработки}

В качестве языка программирования был выбран python3~\cite{python3}, так как его стандартная библиотека достаточна для реализации данных алгоритмов, а также данный язык обладает множеством инструментов для визуализации и работы с данными и таблицами.

В качестве основного файла был выбран инструмент jupyter notebook~\cite{python3-jupyter}, так как он позволяет организовать код в виде блоков, а также выводить данные и графики прямо в нём, что позволяет наглядно продемонстрировать все замеры.

Для построения графиков использовалась библиотека matplotlib~\cite{python3-matplotlib}.

Для замеров процессорного времени использовалась библиотека time~\cite{python3-time}

\section{Реализация алгоритмов}

\lstinputlisting[caption={Классический алгоритм умножения матриц}]{../src/simple.py}

При реализации алгоритма Копперсмита-Винограда были применены следующие оптимизации:
\begin{itemize}
    \item инкремент счётчика наиболее вложенного цикла на 2; 
    \item объединение III и IV частей алгоритма Винограда; 
    \item вынос начальной итерации из каждого внешнего цикла.
\end{itemize}

\

\

\begin{lstlisting}[caption={Алгоритм Копперсмита-Винограда умножения матриц}]
def multiply_optimized(A: list[list[float]], B: list[list[float]]) -> list[list[float]]:
    if len(A[0]) != len(B):
        raise ValueError("Size mismatch")
    n: int = len(B)

    res: list[list[float]] = [
        [
            0. for _ in range(len(B[0]))
        ] for _ in range(len(A))
    ]

    Harr: list[float] = [0. for _ in range(len(A))]
    Varr: list[float] = [0. for _ in range(len(B[0]))]

    for j in range(n // 2):
        Harr[0] = Harr[0] + A[0][2*j] * A[0][2*j + 1]

    for i in range(1, len(A)):
        for j in range(n // 2):
            Harr[i] = Harr[i] + A[i][2*j] * A[i][2*j + 1]

    #

    for j in range(n // 2):
        Varr[0] = Varr[0] + B[2*j][0] * B[2*j + 1][0]

    for i in range(1, len(B[0])):
        for j in range(n // 2):
            Varr[i] = Varr[i] + B[2*j][i] * B[2*j + 1][i]

    #

    for j in range(len(B[0])):
        res[0][j] = -Harr[0] - Varr[j]
        for k in range(0, n - 1, 2):
            res[0][j] = res[0][j] + (A[0][k] + B[k+1][j]) * \
                (A[0][k+1] + B[k][j])
        if n % 2 != 0:
            res[0][j] += A[0][-1] * B[-1][j]

    for i in range(1, len(A)):
        for j in range(len(B[0])):
            res[i][j] = -Harr[i] - Varr[j]
            for k in range(0, n - 1, 2):
                res[i][j] = res[i][j] + (A[i][k] + B[k+1][j]) * \
                    (A[i][k+1] + B[k][j])
            if n % 2 != 0:
                res[i][j] = res[i][j] + A[i][-1] * B[-1][j]

    return res
\end{lstlisting}

\section{Функциональные тесты}

\begin{table}[h!]
\begin{tabular}{|c|c|c|c|c|c|}
\hline
\multirow{2}{*}{A} & \multirow{2}{*}{B} & \multirow{2}{*}{A$\times$B} & \multicolumn{3}{c|}{Результат выполнения алгоритма} \\ \cline{4-6} 
&&& \multicolumn{1}{c|}{Классического} & \multicolumn{1}{c|}{Винограда} & Винограда с опт. \\
\hline
$\begin{pmatrix}
    1 & 2 \\
    3 & 4 \\
    5 & 6
\end{pmatrix}$
&
$
\begin{pmatrix}
    7 & 8 & 9 \\
    0 & 1 & 2
\end{pmatrix}
$
&
$
\begin{pmatrix}
    7 & 10 & 13 \\
    21& 28 & 35 \\
    35& 46 & 57
\end{pmatrix}
$
&
$
\begin{pmatrix}
    7 & 10 & 13 \\
    21& 28 & 35 \\
    35& 46 & 57
\end{pmatrix}
$
&
$
\begin{pmatrix}
    7 & 10 & 13 \\
    21& 28 & 35 \\
    35& 46 & 57
\end{pmatrix}
$
&
$
\begin{pmatrix}
    7 & 10 & 13 \\
    21& 28 & 35 \\
    35& 46 & 57
\end{pmatrix}
$
\\ \hline
$
\begin{pmatrix}
    1 & 2 \\
    3 & 4
\end{pmatrix}
$
&
$
\begin{pmatrix}
    5 & 6 \\
    7 & 8
\end{pmatrix}
$
&
$
\begin{pmatrix}
    19 & 22 \\
    43 & 50
\end{pmatrix}
$
&
$
\begin{pmatrix}
    19 & 22 \\
    43 & 50
\end{pmatrix}
$
&
$
\begin{pmatrix}
    19 & 22 \\
    43 & 50
\end{pmatrix}
$
&
$
\begin{pmatrix}
    19 & 22 \\
    43 & 50
\end{pmatrix}
$
\\ \hline
$
\begin{pmatrix}
    1 & 2
\end{pmatrix}
$
&
$
\begin{pmatrix}
    3 & 4
\end{pmatrix}
$
&
---
&
---
&
---
&
---
\\ \hline
\end{tabular}
\end{table}

Все тесты пройдены успешно

\section*{Вывод}

В ходе работы были реализованы алгоритмы классического умножения матриц и умножения матриц Копперсмита-Винограда, а также было проведено их тестирование.

\clearpage