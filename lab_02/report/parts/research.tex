\chapter{Исследовательская часть}

\section{Технические характеристики}

Технические характеристики устройства, на котором проводились замеры:

\begin{itemize}
	\item операционная система: EndeavourOS x86\_64;
	\item процессор: 13th Gen Intel(R) Core(TM) i53500H (16) С частотой 4.70 ГГц;
	\item оперативная память: 16 ГБ с частотой 5200 МГц.
\end{itemize}

\section{Временные характеристики}

Замеры проводились на квадратных матрицах размера N$\times$N.

Время приведено в миллисекундах

Замеры проводились с использованием библиотеки time~\cite{python3-time}. Замерялось процессорное время.

Результаты замеров приведены на рисунке~\ref{fig:profiling}

\begin{figure}[h!]
\includesvg{profiling.svg}
\caption{Результаты замеров временных затрат алгоритмов}
\label{fig:profiling}
\end{figure}

% \begin{longtable}[]{|c|r|r|r|}
% \caption{Результаты замеров временных затрат алгоритмов}
% \label{tbl:profiling}\\
% \hline
% \multirow{2}{*}{N} & \multicolumn{3}{c|}{Время умножения матриц (мс)} \\ \cline{2-4}
% & \multicolumn{1}{c|}{Классическим алгоритмом} & \multicolumn{1}{c|}{Алгоритмом Винограда} & Алгоритмом Винограда с опт.
% \csvreader{table.csv}{}{\\\hline \csvcoli & \csvcolii & \csvcoliii & \csvcoliv}
% \\ \hline
% \end{longtable}
\section{Вывод}

В результате исследования получен следующий результат: временные затраты на умножение двух квадратных матриц алгоритмом Винограда стабильно ниже временных затрат классического алгоритма умножения таких же матриц. Оптимизированная версия алгоритма Винограда даёт видимый прирост в скорости работы алгоритма. 

\clearpage