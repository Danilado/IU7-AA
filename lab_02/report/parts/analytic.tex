\chapter{Аналитическая часть}

В данном разделе будут рассмотрены следующие алгоритмы умножения матриц:
\begin{itemize}
    \item классический алгоритм;
    \item алгоритм Винограда.
\end{itemize}

\section{Формализация понятия матрицы}

\emph{Матрица} --- это таблица чисел, записанная в форме~(\ref{eq:matrix-def})~\cite{matrix-citation}
\begin{equation}
    \label{eq:matrix-def}
    \begin{pmatrix}
        a_{11} & a_{12} & \dots & a_{1N} \\
        a_{21} & a_{22} & \dots & a_{2N} \\
        \dots & \dots & \ddots & \dots \\
        a_{M1} & a_{M2} & \dots & a_{MN} \\
    \end{pmatrix}
\end{equation}

Числа $M$ и $N$ --- называются размерностями матрицы и определяют количество столбцов и строк соответственно.

Путь $A$ --- матрица. Тогда Обозначение $A_{ij}$ или $a_{ij}$ означает элемент матрицы $A$, расположенный на $i$-й строке и $j$-м столбце.

\section{Определение умножения матриц}

Для данной лабораторной работы также необходимо определить операцию умножения матриц.

Пусть $A$, $B$ и $C$ - матрицы.

Выполнить операцию умножения матриц можно только если число столбцов первой матрицы равно количеству строк второй матрицы. Пусть матрицы $A$ и $B$ удовлетворяют этому условию.

Пусть матрица $C$ задаётся отношением~(\ref{eq:c-matmul})

\begin{equation}
    \label{eq:c-matmul}
    C = \begin{pmatrix}
        c_{i,j}
    \end{pmatrix}, \quad c_{ij} = \sum\limits_{ k=1}^{ N} {B_{ik} \times A{kj}}, \quad i,j = \overline{1..N} 
\end{equation}

Где $N$ - количество столбцов первой матрицы.

Тогда уравнение~(\ref{eq:matmul-def}) есть определение произведения матриц $A$ и $B$~\cite{matrix-citation}

\begin{equation}
    \label{eq:matmul-def}
    C = AB
\end{equation}

\section{Классический алгоритм умножения матриц}

Классический алгоритм умножения матриц является вычислением каждой ячейки выходной матрицы по формуле~(\ref{eq:c-matmul})

\subsection{Преимущества}

Преимуществом данного алгоритма является относительная простота написания - для его работы не требуется дополнительной обработки матриц и подготовки каких-либо данных.

\subsection{Недостатки}

Главным недостатком этого алгоритма является скорость его выполнения.

\section{Алгоритм Копперсмита-Винограда}

Алгоритм Копперсмита-Винограда --- алгоритм, основанный на идее подготовки части данных для более эффективного умножения матриц~\cite{Winograd}. В первом издании (1990 года) авторам удалось достичь асимптотической сложности $O(n^{2.3755})$, где $n$ - размер сторон матрицы.

Идея алгоритма основывается на следующем наблюдении: каждый элемент выходной матрицы является скалярным произведением строки первой и столбца второй матрицы. Скалярное произведение можно преобразовать и использовать преобразованную форму для вычисления произведения матрицы.

Тогда аналогом формулы~(\ref{eq:c-matmul}) для алгоритма Копперсмита-Винограда будет формула~(\ref{eq:wino-matmul})

\begin{equation}
    \label{eq:wino-matmul}
    c_{ij} = \sum\limits_{ k=1}^{ q/2}{(a_{i,2k-1} + b_{2k,j})(a_{i,2k} + b_{2k-1, j})} - \sum\limits_{ k=1}^{ q/2}{(a_{i, 2k-1}\times a_{i,2k}) + (b_{2k-1,j} \times b_{2k, j})}  
\end{equation}

Операнды со знаком из уравнения~(\ref{eq:wino-matmul}) могут быть вычислены заранее для каждого столбца и строки матрицы и переиспользованы в процессе вычисления вычисления выходной матрицы, что позволяет значительно увеличить скорость расчёта.

\subsection{Преимущества}

Главным преимуществом данного алгоритма является скорость выполнения. Особенно заметным прирост становится при матрицах большого размера~\cite{Winograd}.

\subsection{Недостатки}

Недостатками этого алгоритма являются:
\begin{itemize}
    \item необходимость в предварительной обработке матриц;
    \item необходимость в выделении дополнительной памяти для хранения рассчитанных значений;
    \item дополнительная обработка тех случаев, когда количество строк и столбцов в матрице является нечётным. 
\end{itemize}

\section{Вывод}

Были рассмотрены классический алгоритм умножения матриц, и алгоритм \\Копперсмита-Винограда.