\ssr{ВВЕДЕНИЕ}

\textbf{Графовые модели} позволяют представлять алгоритмы и их выполнение в виде графов~\cite{lectures}.

Таким образом, графовой моделью называется конечный орграф, вершины которого являются некоторыми командами или участками исполняемого кода, а дуги выражают некоторое отношение между этими участками.
С помощью графовой модели могут быть выражены отношения последовательности, зависимости по данным и другие.

Графовые модели могут использоваться для поиска участков программы, которые могут быть выполнены параллельно.

Целью данной работы является построение четырёх графовых моделей для фрагмента кода, а также определение применимости графовых моделей к задаче анализа программного кода.

Для достижения этой цели нужно выполнить следующие задачи:
\begin{itemize}
  \item выбрать фрагмент кода из лабораторной номер 5;
  \item построить графовые модели;
  \item сделать вывод о применимости графовых моделей к задаче анализа программного кода.
\end{itemize}

\clearpage