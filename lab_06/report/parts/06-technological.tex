\section{Технологическая часть}

\subsection{Требования к программному обеспечению}\label{section:requirements}
К программе предъявлен ряд требований: 
\begin{itemize}
    \item входные данные --- матрица расстояний между городами;
    \item выходные данные --- наилучший маршрут, найденный алгоритмами.
\end{itemize}

\subsection{Средства реализации}

Для реализации данной лабораторной работы был выбран язык C++ стандарта C++20~\cite{cpp}. Выбор сделан в пользу данного языка в связи с его высоким быстродействием, в связи с наличием у него встроенного модуля $ctime$ для измерения процессорного времени, а также из-за наличия всех необходимых структур данных для реализации алгоритмов.
 
\clearpage
\subsection{Реализация алгоритмов}

Реализация алгоритма полного перебора изображена на листинге~\ref{lst:full}

Реализация муравьиного алгоритма перебора изображена на листинге~\ref{lst:ant}

\begin{lstlisting}[caption={Реализация алгоритма полного перебора},label=lst:full]
#include "FullPF.hpp"

#include <algorithm>
#include <limits>
#include <ranges>

std::pair<double, std::vector<int>>
FullCombinationPathFinder::find(const Matrix &mat) {
    std::vector<int> places(mat.size());
    for (const int &i : std::views::iota(0ull, mat.size()))
    places[i] = i;

    std::vector<std::vector<int>> placesCombinations;
    while (std::next_permutation(places.begin(), places.end()))
    placesCombinations.push_back(places);

    double minDist = std::numeric_limits<double>::infinity();
    std::vector<int> bestWay;

    for (const auto &combination : placesCombinations) {
    double curDist = 0;
    for (const int &j : std::views::iota(0ull, mat.size() - 1)) {
        int startCity = combination[j];
        int endCity = combination[j + 1];
        curDist += mat[startCity][endCity];
    }

    curDist += mat[combination[mat.size() - 1]][combination[0]];

    if (curDist < minDist) {
        minDist = curDist;
        bestWay = combination;
        bestWay.push_back(bestWay[0]);
    }
    }

    return {minDist, bestWay};
}
\end{lstlisting}

\begin{lstlisting}[caption={Реализация муравьиного алгоритма},label=lst:ant]
#include "AntPF.hpp"

#include <algorithm>
#include <cmath>
#include <iostream>
#include <limits>
#include <random>
#include <ranges>
#include <vector>

const double MIN_PHEROMONE = 1e-10;

std::pair<double, std::vector<int>> AntPathFinder::find(const Matrix &mat) {
    double q = calc_q(mat);
    std::vector<int> best_way;
    double min_length = std::numeric_limits<double>::infinity();

    Matrix pheromones = get_pheromones(mat);
    Matrix visibility = get_visibility(mat);
    int ants = mat.size();

    for (const auto &_ : std::views::iota(0, m_days)) {
    std::vector<int> route(mat.size());
    std::iota(route.begin(), route.end(), 0);

    std::vector<std::vector<int>> visited = get_visited_places(route, ants);

    for (const auto &ant : std::views::iota(0, ants)) {
        while (visited[ant].size() != mat.size()) {
        std::vector<double> pk =
            find_possibilities(pheromones, visibility, visited, ant, mat);
        int chosen_place = choose_next(pk);
        visited[ant].push_back(chosen_place);
        }

        visited[ant].push_back(visited[ant][0]);

        double cur_length = calc_length(visited[ant], mat);
        if (cur_length < min_length) {
        min_length = cur_length;
        best_way = visited[ant];
        }
    }

    pheromones = update_pheromones(visited, pheromones, q, mat);
    pheromones = update_pheromones_elite(pheromones, q, best_way, mat);
    }

    return {min_length, best_way};
}

double AntPathFinder::calc_q(const Matrix &mat) {
    double q = 0;
    int count = 0;

    for (const auto &i : std::views::iota(0ull, mat.size()))
    for (const auto &j :
            std::views::iota(0ull, mat.size()) |
                std::views::filter([&i](int j) { return i != j; })) {
        q += mat[i][j];
        count++;
    }

    return q / count;
}

AntPathFinder::Matrix AntPathFinder::get_pheromones(const Matrix &mat) {
    double min_phero = 1.0;
    return Matrix(mat.size(), std::vector<double>(mat.size(), min_phero));
}

AntPathFinder::Matrix AntPathFinder::get_visibility(const Matrix &mat) {
    Matrix visibility(mat.size(), std::vector<double>(mat.size(), 0.));

    for (const auto &i : std::views::iota(0ull, mat.size()))
    for (const auto &j : std::views::iota(0ull, mat.size()) |
                                std::views::filter([&i](int j) { return i != j; }))
        visibility[i][j] = 1.0 / mat[i][j];

    return visibility;
}

std::vector<std::vector<int>>
AntPathFinder::get_visited_places(const std::vector<int> &route, int ants) {
    std::vector<std::vector<int>> visited(ants);
    for (const auto &ant : std::views::iota(0, ants))
    visited[ant].push_back(route[ant]);

    return visited;
}

double AntPathFinder::calc_length(const std::vector<int> &route,
                                    const Matrix &mat) {
    double length = 0;
    for (size_t way_len = 1; way_len < route.size(); ++way_len) {
    length += mat[route[way_len - 1]][route[way_len]];
    }
    return length;
}

AntPathFinder::Matrix
AntPathFinder::update_pheromones(const std::vector<std::vector<int>> &visited,
                                    Matrix &pheromones, double q,
                                    const Matrix &mat) {
    for (const auto &i : std::views::iota(0ull, mat.size()))
    for (const auto &j : std::views::iota(0ull, mat.size())) {
        double delta_pheromones = 0;
        for (int ant = 0; ant < mat.size(); ++ant) {
        double length = calc_length(visited[ant], mat);
        delta_pheromones += q / length;
        }
        pheromones[i][j] *= (1 - m_k_evaporation);
        pheromones[i][j] += delta_pheromones;

        if (pheromones[i][j] < MIN_PHEROMONE) {
        pheromones[i][j] = MIN_PHEROMONE;
        }
    }

    return pheromones;
}

AntPathFinder::Matrix
AntPathFinder::update_pheromones_elite(Matrix &pheromones, double q,
                                        const std::vector<int> &best_way,
                                        const Matrix &mat) {
    double length = calc_length(best_way, mat);
    for (const auto &i : std::views::iota(0ull, mat.size()))
    for (const auto &j : std::views::iota(0ull, mat.size())) {
        double delta_pheromones = m_e * q / length;
        pheromones[i][j] += delta_pheromones;
    }

    return pheromones;
}

std::vector<double> AntPathFinder::find_possibilities(
    const Matrix &pheromones, const Matrix &visibility,
    const std::vector<std::vector<int>> &visited, int ant, const Matrix &mat) {
    std::vector<double> pk(mat.size(), 0.);
    int ant_place = visited[ant].back();

    for (const auto &place : std::views::iota(0ull, mat.size())) {
    if (std::find(visited[ant].begin(), visited[ant].end(), place) ==
        visited[ant].end()) {
        pk[place] = std::pow(pheromones[ant_place][place], m_alpha) *
                    std::pow(visibility[ant_place][place], m_beta);
    }
    }

    if (visited[ant].size() == mat.size()) {
    int start_place = visited[ant][0];
    pk[start_place] = std::pow(pheromones[ant_place][start_place], m_alpha) *
                        std::pow(visibility[ant_place][start_place], m_beta);
    }

    double sum_pk = std::accumulate(pk.begin(), pk.end(), 0.0);
    for (const auto &place : std::views::iota(0ull, mat.size()))
    pk[place] /= sum_pk;

    return pk;
}

int AntPathFinder::choose_next(const std::vector<double> &pk) {
    double possibility = static_cast<double>(rand()) / RAND_MAX;
    double choice = 0;
    int chosen_place = 0;

    while ((choice < possibility) && (chosen_place < pk.size())) {
    choice += pk[chosen_place];
    chosen_place++;
    }

    return chosen_place - 1;
}
    
\end{lstlisting}


\clearpage
\subsection{Функциональные тесты}

В таблице~\ref{tbl:func-tests} приведены функциональные тесты для алгоритмов. 
Все тесты пройдены успешно.

В таблице используются следующие обозначения.
\begin{enumerate}
    \item Длина пути, полученная методом полного перебора --- МПП.
    \item Длина пути, полученная методом на основе муравьиного алгоритма --- ММА.
    \item Оптимальный результат (длина пути) --- ОР.
\end{enumerate}

\begin{table}[ht]
    \begin{center}
        \caption{Функциональные тесты}
        \label{tbl:func-tests}
        \begin{tabular}{|c|c|c|c|c|}
            \hline
            Матрица смежности & ОР & МПП & ММА \\
            \hline
            $ \begin{pmatrix}
                0 & 3 & 8 & 7 & 10 \\
                3 & 0 & 7 & 5 & 4 \\
                8 & 7 & 0 & 5 & 5 \\
                7 & 5 & 5 & 0 & 3 \\
                10 & 4 & 5 & 3 & 0 
            \end{pmatrix}$ &
            23 &
            23 & 23 \\
            \hline
            $ \begin{pmatrix}
                0 & 3 & 4 \\
                3 & 0 & 5 \\
                4 & 5 & 0	
            \end{pmatrix}$ &
            12  &
            12 & 12  \\
            \hline
            $ \begin{pmatrix}
                0 & 27 & 12 & 21 \\
                27 & 0 & 16 & 20 \\
                12 & 16 & 0 & 11 \\
                21 & 20 & 11 & 0 
            \end{pmatrix}$ &
            69 &
            69 & 69  \\
            \hline
        \end{tabular}
    \end{center}
    \end{table}

Получение оптимального результата для муравьиного алгоритма не гарантировано, поэтому тест считается пройденным реализацией муравьиного алгоритма, если результат входит в лучшие 10\%, то есть $\text{OP}*1.1$.