\section{Аналитическая часть}

\subsection{Задача коммивояжера}

Обычно задачу коммивояжера формулируют следующим образом~\cite{applegate2011traveling}: Пусть дан список городов и расстояний между каждой парой городов, необходимо найти кратчайший маршрут, в котором будет единожды посещён каждый город и вернётся в исходную точку.

Данная формулировка подразумевает возврат в исходный город, то есть поиск ``Гамильтонова'' пути. Существуют и формулировки, не требующие точного возврата в исходную точку.

Существует ряд методов для решения этой задачи. В данной работе будут рассмотрены метод полного перебора и метод с использованием ``Муравьиного'' (роевого) алгоритма.

\subsection{Метод полного перебора}

При использовании данного метода решение задачи получается путем перебора всех возможных подходящих путей и выбор наименьшего из них, путём поиска наименьшего суммарного расстояния.

Преимущество данного метода заключается в том, полученный результат гарантированно является оптимальным решением.

Главный недостатком этого метода является его алгоритмическая сложность. Полный перебор подразумевает сложность --- О($n!$), где $n$ --- количество вершин графа городов~\cite{ylanov}.

\subsection{Метод на основе муравьиного алгоритма}
Данный метод основан на имитации природных механизмов самоорганизации муравьев~\cite{antAlgs}.

Если наблюдать за муравьями, можно увидеть, что они почти всегда ходят по конкретным путям, и эти пути зачастую являются довольно эффективными.

В природе муравьи используют свои органы чувств для ориентации в пространстве, а также феромон для непрямого обмена информацией друг с другом.

Имитация поведения колонии муравьёв позволяет эффективно решать задачи для поиска путей на графах, так как сам феромон и его свойство --- испарение, позволяют не задерживаться в локальных экстремумах для комбинаторных задач или обходить разные пути как, например, в задаче коммивояжера.

\subsubsection{Математическая модель метода на основе муравьиного алгоритма}
Пусть муравей имеет следующие характеристики-чувства:
\begin{enumerate}[label=\arabic*)]
	\item зрение --- способность определить длину ребра;
	\item обоняние --- способность чуять феромон;
	\item память --- способность запомнить пройденный маршрут.
\end{enumerate}

Когда перед муравьём встаёт задача выбора одного ребра графа из нескольких возможных, он ориентируется на свои чувства.

Для нашей задачи, запомнив посещённые вершины графа, муравей не станет выбирать их.

Ориентируясь на длину ребра с помощью зрения, посредством функции~\eqref{d_func}, определяется начальная привлекательность ребра, в зависимости от его длины (значения метки ребра в матрице смежности):
\begin{equation}
	\label{d_func}
	\eta_{ij} = 1 / D_{ij},
\end{equation}
где $D_{ij}$ — расстояние от текущей вершины $i$ до заданной вершины $j$.

Как видно из вершины, привлекательность ребра обратно пропорциональна его длине~\cite{antAlgs}.

Вероятность перехода муравья в конкретную вершину графа на основе всех чувств выражается формулой~\eqref{posib}.
\begin{equation}
	\label{posib}
	p_{k,ij} = \begin{cases}
		\frac{\eta_{ij}^{\alpha}\cdot\tau_{ij}^{\beta}}{\sum_{q\notin J_k} \eta^\alpha_{iq}\cdot\tau^\beta_{iq}}, j \notin J_k \\
		0, j \in J_k
	\end{cases}
\end{equation}
где $\alpha$ --- влияния длины пути, $\beta$ --- влияния феромона, $\tau_{ij}$ --- количество феромона на ребре $ij$, $\eta_{ij}$ --- привлекательность ребра $ij$ на основе его длины, $J_k$ --- список посещенных за текущий день вершин. При этом коэффициенты $\alpha$ и $\beta$ связаны отношением $\beta = 1 - \alpha$.

Муравьиный алгоритм работает, симулируя несколько ``дней'' работы колонии. 

Каждый ``день'' каждый муравей проходит по одному пути.

Ночью, перед наступлением очередного дня (после завершения работы всех муравьев), значение феромона обновляется по формуле \eqref{update_phero_1}.
\begin{equation}
	\label{update_phero_1}
	\tau_{ij}(t+1) = \tau_{ij}(t)\cdot(1-p) + \Delta \tau_{ij}(t).
\end{equation}

При этом
\begin{equation}
	\label{update_phero_2}
	\Delta \tau_{ij}(t) = \sum_{k=1}^N \Delta \tau^k_{ij}(t),
\end{equation}
где
\begin{equation}
	\label{update_phero_3}
	\Delta\tau^k_{ij}(t) = \begin{cases}
		Q/L_{k}, \textrm{ребро посещено муравьем $k$ в текущий день $t$,} \\
		0, \textrm{иначе,}
	\end{cases}
\end{equation}
где $Q$ --- ``квота'', общее количество феромона, выделяемое муравьём. Значение $Q$ равно средней длине ребра графа (или средней метке вершин графа). $L_k$ --- общая длина пути, пройденного муравьём $k$ в этот день.

Так как при переходах между вершинами используется вероятность~\eqref{posib}, то необходимо проверить, чтобы она не стала нулевой, а именно ввести значение $\tau_{min}$ --- минимально возможное значение феромона, и не допускать уменьшение значения феромона ниже $\tau_{min}$.

В итоге путь муравья выбирается по следующему алгоритму:
\begin{enumerate}
	\item каждый муравей имеет список запретов --- список уже посещенных вершин графа;
	\item зрение отвечает за эвристическое желание посетить вершину;
	\item обоняние отвечает за ощущение феромона на определенном ребре. При этом количество феромона на ребре в день $t$ обозначается как $\tau_{i, j} (t)$;
	\item при прохождение ребра муравей откладывает на нем феромон, который показывает оптимальность выбора данного ребра, его количество вычисляется по формуле~\eqref{update_phero_3}.
\end{enumerate}

Следует отметить, что обновление значения феромона происходит только в конце очередного для, то есть если в течение одного дня муравей один муравей прошёл по определённому пути, значение феромона на этом пути не будет обновлено до окончания этого дня.

Если все муравьи начинают свой путь в одной конкретной вершине, то решение задачи может потерять в точности, так как оптимальные пути выхода с этой вершины будут получать всё больше и больше феромона. Согласно постановке задачи, вершина начала пути не определена, то есть может быть выбрана любая из доступных.

Также существуют вариации муравьиного алгоритма, значительно увеличивающие его эффективность.

В этой работе рассматривается вариация с ``элитными'' муравьями. Элитные муравьи проходят свой путь только по наилучшему маршруту, найденному в алгоритме на данной итерации.

Такое поведение искусственно добавляет дополнительный феромон на рёбра лучшего маршрута, найденного на данный момент.

Такой алгоритм может получить решения, отличающиеся от того, что получит обычный муравьиный алгоритм.
