\chapter{Конструкторская часть}

\section{Требования к программному обеспечению}

К разрабатываемой программе предъявлен ряд требований:

\begin{itemize}
	\item на вход подаются массив arr и искомый элемент x;
	\item на выход подаётся целое число или сообщение об ошибке;
	\item индексация в массиве начинается с нуля;
	\item над элементами массива определено отношение порядка;
	\item если искомый элемент не находится в массиве, алгоритмы возвращают -1 в качестве индекса;
	\item должна быть возможность вывода графиков количества сравнений в зависимости от положения элемента в массиве.
\end{itemize}

\section{Разработка алгоритмов}

Алгоритм простого поиска изображён на рисунке~\ref{fig:alg_simple}

Алгоритм двоичного поиска изображён на рисунке~\ref{fig:alg_bin}

\clearpage

\begin{figure}[h!]
	\centering
	\includesvg[height=0.9\textheight]{alg_simple}
	\caption{Алгоритм простого поиска элемента в массиве}
	\label{fig:alg_simple}
\end{figure}

\begin{figure}[h!]
	\centering
	\includesvg[height=0.9\textheight]{alg_binary}
	\caption{Алгоритм двоичного поиска элемента в массиве}
	\label{fig:alg_bin}
\end{figure} 

\section{Вывод}

В результате конструкторской части были определены требования к ПО, а также построены схемы алгоритмов простого и двоичного поиска элементов в массиве.

\clearpage