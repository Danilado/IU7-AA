\ssr{ЗАКЛЮЧЕНИЕ}

Была проанализирована трудоёмкость алгоритмов поиска в массиве.

В результате исследования установлено, что алгоритм двоичного поиска является более эффективным, чем алгоритм простого поиска для отсортированных массивов длины равной 1022 элементам.

Среднее количество сравнений для простого и двоичного алгоритмов поиска равно 511 и 17 сравнениям соответственно. 

Максимальное количество сравнений для простого и двоичного алгоритмов поиска равно 1022 и 19 сравнениям соответственно.

Были выполнены следующие задачи:
\begin{itemize}
  \item были рассмотрены существующие алгоритмы поиска элемента в массиве;
  \item рассмотренные алгоритмы поиска в массиве были разработаны;
  \item разработанные алгоритмы были реализованы;
  \item полученные реализации были проанализированы по трудоёмкости.
\end{itemize}

Цели и задачи лабораторной работы выполнены.
