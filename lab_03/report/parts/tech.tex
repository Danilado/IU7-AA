\chapter{Технологическая часть}
\section{Средства разработки}

В качестве языка программирования был выбран python3~\cite{python3}, так как его стандартная библиотека достаточна для реализации данных алгоритмов, а также данный язык обладает множеством инструментов для визуализации и работы с данными и таблицами.

В качестве основного файла был выбран инструмент jupyter notebook~\cite{python3-jupyter}, так как он позволяет организовать код в виде блоков, а также выводить данные и графики прямо в нём, что позволяет наглядно продемонстрировать все замеры.

Для построения графиков использовалась библиотека matplotlib~\cite{python3-matplotlib}.

\section{Реализация алгоритмов}

\begin{lstlisting}[label=simple-search,caption={Простой алгоритм поиска элемента в массиве}]
def search(arr: list[int], x: int) -> int:
  for i in range(len(arr)):
      if arr[i] == x:
          return i
  return -1
\end{lstlisting}

\begin{lstlisting}[label=bin-search,caption={Бинарный алгоритм поиска элемента в массиве}]
def binsearch(arr: list[int], x: int) -> int:
  left = 0
  right = len(arr) - 1

  while left <= right:
      c = (right + left) // 2
      if arr[c] == x:
          return c
      elif arr[c] < x:
          left = c + 1
      else:
          right = c - 1

  return -1
\end{lstlisting}

Для замеров количества сравнений, были использованы модифицированные версии алгоритмов, которые подсчитывают сравнения в процессе работы.

\section{Функциональные тесты}

Для функционального тестирования были написаны специальные программы, генерирующие массивы и проверяющие результат работы алгоритма с настоящим положением элемента в массиве.

Эти тесты включают в себя:
\begin{itemize}
  \item тест с элементом, не принадлежащим массиву (листинг~\ref{lst:test_nonexistent});
  \item тест с поиском элемента в пустом массиве (листинг~\ref{lst:test_empty});
  \item тесты с поиском существующего в массиве элемента (листинг~\ref{lst:test_search}).
\end{itemize}

\begin{lstlisting}[label=lst:test_nonexistent,caption={Тест поиска несуществующего элемента в массиве}]
arr = [i for i in range(N)]

res = simple.search(arr, -1)
if (res == -1):
    print("Nonexistent element test passed")
else:
    print("Nonexistent element test failed")

res = binary.binsearch(arr, -1)
if (res == -1):
    print("Nonexistent element test passed")
else:
    print("Nonexistent element test failed")
\end{lstlisting}

\begin{lstlisting}[label=lst:test_empty,caption={Тест поиска элемента в пустом массиве}]
res = simple.search([], 0)
if (res == -1):
    print("Empty array test passed")
else:
    print("Empty array test failed")

res = binary.binsearch([], 0)
if (res == -1):
    print("Empty array test passed")
else:
    print("Empty array test failed")
\end{lstlisting}

\begin{lstlisting}[label=lst:test_search,caption={Тест поиска элементов в массиве}]
arr = [i for i in range(N)]

for i in range(N):
    print(f"{i=}", end="\r")
    res = simple.search(arr, i)
    if res != i:
        print(f"Simple search fail {i=}")

for i in range(N):
    print(f"{i=}", end="\r")
    res = binary.binsearch(arr, i)
    if res != i:
        print(f"Binary search fail {i=}")
\end{lstlisting}

Все тесты пройдены успешно

\section*{Вывод}

В ходе работы были разработаны алгоритмы простого и бинарного поиска элемента в массиве, а также было проведено их тестирование.

\clearpage