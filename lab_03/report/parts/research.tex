\chapter{Исследовательская часть}

\section{Технические характеристики}

Технические характеристики устройства, на котором проводились замеры:

\begin{itemize}
	\item операционная система: EndeavourOS x86\_64;
	\item процессор: 13th Gen Intel(R) Core(TM) i53500H (16) С частотой 4.70 ГГц;
	\item оперативная память: 16 ГБ с частотой 5200 МГц.
\end{itemize}

\section{Исследование количества сравнений}

В ходе исследования проводились замеры количества количества сравнений, необходимого для нахождения элемента в массиве.

Для измерения используются отсортированные массивы из 1022 элементов, где индекс элемента в массиве совпадает с его значением.

На рисунке~\ref{fig:simple_graph} изображён график зависимости количества сравнений при поиске элемента в массиве от индекса этого элемента с помощью алгоритма простого поиска

На рисунке~\ref{fig:binary_graph} изображена гистограмма, отображающая аналогичную зависимость для алгоритма бинарного поиска

На рисунке~\ref{fig:binary_graph_sorted} изображена гистограмма из рисунка~\ref{fig:binary_graph}, отсортированная по количеству сравнений

\begin{figure}[h!]
	\includesvg[width=0.8\textwidth]{simple_graph.svg}
	\caption{График зависимости количества сравнений от индекса элемента с помощью алгоритма простого поиска}
	\label{fig:simple_graph}
\end{figure}

\begin{figure}[h!]
	\includesvg[width=0.9\textwidth]{binary_graph.svg}
	\caption{Гистограмма зависимости количества сравнений от индекса элемента с помощью алгоритма двоичного поиска}
	\label{fig:binary_graph}
\end{figure}

\begin{figure}[h!]
	\includesvg[width=0.9\textwidth]{binary_graph_sorted.svg}
	\caption{Гистограмма зависимости количества сравнений от индекса элемента с помощью алгоритма двоичного поиска, отсортированная по количеству сравнений}
	\label{fig:binary_graph_sorted}
\end{figure}

\section{Вывод}

В результате измерений было обнаружено, что при размере массива равном 1022 элементам, среднее количество сравнений для простого и двоичного алгоритмов поиска равно 511 и 17 сравнениям соответственно. 

Максимальное количество сравнений для простого и двоичного алгоритмов поиска равно 1022 и 19 сравнениям соответственно.

В результате исследования установлено, что алгоритм двоичного поиска является более эффективным, чем алгоритм простого поиска для отсортированных массивов длины равной 1022 элементам. 

\clearpage