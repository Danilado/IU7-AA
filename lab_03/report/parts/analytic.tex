\chapter{Аналитическая часть}

Существует два алгоритма поиска элемента в массиве~\cite{search-citation}:
\begin{itemize}
    \item алгоритм полного перебора;
    \item алгоритм бинарного поиска.
\end{itemize}

\section{Алгоритм полного перебора}

\emph{Алгоритм полного перебора} (также называемый простым алгоритмом поиска или линейного поиска) --- это алгоритм поиска элемента в массиве, подразумевающий последовательное итерирование по всем элементам в массиве. Во время каждой итерации очередной элемент массива сравнивается с искомым. Если значения совпали, итерирование прекращается и алгоритм возвращает индекс текущего элемента.

\subsection{Преимущества}

Преимуществом этого алгоритма является полное отсутствие каких-либо ограничений на содержимое массива. (не важен порядок элементов или тип данных, если можно определить равенство объектов этого типа)

\subsection{Недостатки}

Недостатком этого алгоритма является скорость его выполнения --- в зависимости от того, на какой позиции расположен элемент в массиве, алгоритму может потребоваться от 1 до $N$ итераций, где $N$ --- количество элементов в массиве.

\section{Алгоритм бинарного поиска}

\emph{Алгоритм бинарного поиска} (также называемый алгоритмом двоичного поиска или поиском делением пополам) --- это алгоритм, выполняющий поиск элемента в отсортированном массиве. Во время каждой итерации, размер той части массива, в который выполняется поиск, делится пополам. Это достигается выбором элемента, стоящего в середине этой части и сравнением его с искомым. Если искомый элемент найден, возвращается индекс этого элемента. Иначе, если искомый элемент больше, отбрасывается ``левая'' часть массива (та, что находится левее текущего элемента), иначе --- ``правая'' (та, что находится правее)

\subsection{Преимущества}

Главным преимуществом этого алгоритма является скорость его выполнения.

\subsection{Недостатки}

Недостатками этого алгоритма являются:
\begin{enumerate}
    \item необходимость в сортировке исходного массива;
    \item необходимость наличия отношения порядка между элементами исходного массива.
\end{enumerate}

\section{Вывод}

Были рассмотрены существующие алгоритмы поиска элемента в массиве. 