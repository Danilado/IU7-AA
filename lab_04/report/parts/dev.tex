\chapter{Разработка ПО}

Для реализации алгоритмов был выбран язык программирования c++ стандарта 14882 (c++20)~\cite{cpp}. Этот язык достаточен для выполнения работы, так как его стандартная библиотека предлагает средства для работы с нативными потоками.

Для доступа к интернет ресурсам была использована библиотека libcurl~\cite{curl}. Она предоставляет интерфейс для работы с интернет запросами, а также предоставляет средства для получения и расшифровки ответов.

Поскольку libcurl --- изначально библиотека для языка Си, приходится самостоятельно создавать объект CURL и освобождать из под него память. Для гарантии отсутствия утечек и упрощённой загрузки страниц, был написан класс-обёртка над библиотекой libcurl, интерфейс, которого представлен в листинге~\ref{lst:curlwrapperdef}, а реализация --- в листинге~\ref{lst:curlwrapperimpl}.

\lstinputlisting[caption={Интерфейс класса-обёртки над библиотекой libcurl}, label={lst:curlwrapperdef}]{../src/curlwrap.hpp}

\lstinputlisting[caption={Реализация класса-обёртки над библиотекой libcurl}, label={lst:curlwrapperimpl}]{../src/curlwrap.cpp}

Следует заметить, что отдельный объект CURL предназначен для выполнения одного запроса в один момент времени, поэтому использовать один объект класса CurlWrapper в нескольких потоках не получится.

Для реализации параллельного выполнения запросов, был написан класс, который распределяет задачи между несколькими потоками. Его интерфейс и реализация представлены на листингах~\ref{lst:threadpooldef} и~\ref{lst:threadpoolimpl} соответственно.


\lstinputlisting[caption={Интерфейс класса-обёртки для параллельного вызова}, label={lst:threadpooldef}]{../src/threadpool.hpp}

\lstinputlisting[caption={Реализация класса-обёртки для параллельного вызова}, label={lst:threadpoolimpl}]{../src/threadpool.cpp}

Используя эти классы, было написано приложение, которые принимает с клавиатуры количество потоков и запросов, которые необходимо совершить.

Полученные ответы запросов записываются в файлы i.html, где i --- порядковый номер запроса, а сами файлы попадают в директорию out.

Дальнейшая обработка файлов будет проводиться в лабораторной работе номер 5.